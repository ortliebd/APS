\section{Introdução} \label{sec:introducao}
Um algoritmo é qualquer procedimento computacional bem definido que toma algum valor ou conjunto de valores como entrada e produz algum valor ou conjunto valores como saída \cite{cormen:01}.

O problema de ordenação pode ser descrito como uma sequência de entrada $\langle a_1, a_2, \ldots, a_n\rangle$ que após a permutação dos elementos gera uma saída $\langle a_1^{'}, a_2^{'}, \ldots, a_n^{'}\rangle$ tal que $a_1^{'}\leq a_2^{'}\leq, \ldots, \leq a_n^{'}$ \cite{cormen:01}.

O problema de encontrar um subvetor máximo pode ser descrito como, dado um vetor com $n$ números inteiros (positivos e negativos) $V[1...n]$, encontre $V[e...d]$ tal que $e \geq 1$ e $d \leq n$ de maneira que $\sum_{i=e}^{d}V[i]$ é máxima \cite{foleiss:01}.

O problema de busca consiste em encontrar um valor, dado um vetor, cujo elemento seja igual a um valor informado.

Analisar a eficiência dos algoritmos torna-se necessário para compreender os recursos computacionais e o tempo gastos para sua execução, alguns problemas apresentam diversas soluções, a análise permite a escolha do algoritmo mais apropriada para determinada aplicação.

\subsection{Objetivos}
O objetivo principal deste trabalho é analisar empiricamente o tempo de execução de diversos algoritmos de ordenação e comparar o desempenho entre eles, bem como analisar o desempenho do algoritmo de busca binária e do algoritmo para encontrar o subvetor de soma máxima.

%%% Local Variables:
%%% TeX-master: "Artigo"
%%% End: